\documentclass[./../main.tex]{subfiles}
\graphicspath{{img/}}

\begin{document}
    \begin{exercise}
        ¿Son posibles los siguientes decaimientos e interacciones?

        \begin{itemize}
            \item \ch{\Omega- -> \Sigma+ + \e- + {$\overline{\nu}$}_{\e}}
            \item \ch{p + \e- -> n + \nu_{\e}}
            \item \ch{\pi+ + n -> \pi+ + p}
        \end{itemize}

        Justifica tus respuestas.

        \begin{solution}
            \begin{itemize}
                \item \ch{\Omega- -> \Sigma+ + \e- + {$\overline{\nu}$}_{\e}}
                
                Para determinar si el decaimiento es posible empezamos verificando si se conserva la energía:

                \begin{align*}
                    \ch{\(\qty{1672.5}{\MeV}\) &-> \(\qty{1184.37}{\MeV}\) + \(\qty{0.511}{\MeV}\) + 0},\\
                    \ch{\(\qty{1672.5}{\MeV}\) &-> \(\qty{1184.881}{\MeV}\)}.
                \end{align*}

                Del lado izquierdo tenemos que la energía es aún mayor, aunque no por tanto, que la energía del lado derecho. Por lo tanto, \setulcolor{pinkwave}\ul{la energía se conserva}.

                Ahora, revisamos si se conserva la carga. Respectivamente tenemos que las cargas correspondientes a cada partícula son \(-1\e\), \(+1\e\), \(-1e\) y \(0\e\), \idest

                \begin{align*}
                    \ch{\(-1\e\) &-> \(1\e\) + \(-1\e\)},\\
                    \ch{\(-1\e\) &-> 0}.
                \end{align*}

                La carga eléctrica es distinta en ambos lados de la expresión, por lo tanto \setulcolor{customBlue}\ul{la carga no se convserva} y puesto que basta únicamente con esto para que la interacción no sea posible, ya no verificamos si la conservación del número bariónico y la del número leptónico se cumplen. Es decir, \setulcolor{pinkwave}\ul{el decaimiento no es posible}.
                
                \item \ch{p + \e- -> n + \nu_{\e}}

                Para facilitar el análisis primero identificaremos el tipo de partículas presentes en la interacción. 
                Tenemos entonces dos bariones, \ch{p} y \ch{n}, y dos leptones, \ch{\e-} y \ch{\nu_{\e}}.

                Nuevamente nos encontramos ante una interacción, \idest una colisión, por lo que omitimos la verificación de la conservación de la energía y únicamente nos toca hacerlo para la carga, el número bariónico y leptónico. Pero inmediatamente nos podemos dar cuenta que tanto del lado izquierdo como del derecho tenemos un barión y un leptón (de la misma familia), todos partículas, tal que \setulcolor{customBlue}\ul{el número bariónico y leptónico se conservan}.

                Finalmente, analizamos la conservación de la carga. Del lado izquierdo tenemos que la carga para el protón es \(+1\e\) y para el electrón es \(-1\e\), que es 0 y, del lado derecho, la carga de ambas partículas es cero. Por lo tanto, \setulcolor{customBlue}\ul{la carga se conserva} y \setulcolor{pinkwave}\ul{la interacción es posible}.

                Veamos entonces su diagrama de Feynman:

                \begin{figure}[htb]
                    \centering
                    \begin{tikzpicture}
                        \begin{feynman}
                            % u -> u
                            \vertex (q1) {\(u\)};
                            \vertex [above right=of q1] (a);
                            \vertex [right=of a] (q2) {\(u\)};
                            % d -> d
                            \vertex [below=0.5cm of q1] (q3) {\(d\)};
                            \vertex [above right=of q3] (b);
                            \vertex [right=of b] (q4) {\(d\)};
                            % u -> d
                            \vertex [below=0.5cm of q3] (q5) {\(u\)};
                            \vertex [above right=of q5] (c);
                            \vertex [right=of c] (q6) {\(d\)};
                            % Weak interaction
                            \vertex [below right=4em of c] (d);
                            % electron and electronic neutrino
                            \vertex [below left=of d] (f1) {\(\e^{-}\)};
                            \vertex [right=of d] (f2) {\(\nu_{\e}\)};

                            \diagram* {
                                (q1) -- [fermion] (a) -- [fermion] (q2),
                                (q3) -- [fermion] (b) -- [fermion] (q4),
                                (q5) -- [fermion] (c) -- [fermion] (q6),
                                (c) -- [scalar, edge label'=\(W^{+}\)] (d),
                                (d) -- [anti fermion] (f1),
                                (d) -- [fermion] (f2),
                            };
                        \end{feynman}
                    \end{tikzpicture}                        
                    \caption{Diagrama de Feynman para la colisión entre un protón y un electrón.}
                    \label{fig:proton-electron-interaction}
                \end{figure}
                
                \item \ch{\pi+ + n -> \pi+ + p}
                
                Sabemos que para este caso únicamente debemos verificar si la carga, el número bariónico y número leptónico se conservan. Notemos entonces que tenemos mesones (\ch{\pi+}) y bariones (\ch{p}, \ch{n}), inmediatamente sabemos que el número leptónico es 0 y que el número bariónico es 1 en ambos lados, tal que \setulcolor{customBlue}\ul{el número bariónico y el número leptónico se conservan}.

                Por otro lado sabemos que la carga del neutrón es 0, dando como resultado que \setulcolor{customBlue}\ul{la carga no se convserva}, \idest

                \begin{align*}
                    \ch{\(+1\e\) &-> \(+1\e\) + \(+1\e\)},\\
                    \ch{\(+1\e\) &-> \(+2\e\)}.
                \end{align*}

                Por lo tanto, \setulcolor{pinkwave}\ul{la interacción no es posible.}
            \end{itemize}
        \end{solution}
    \end{exercise}
\end{document}